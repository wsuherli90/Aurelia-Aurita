% ======================================================================
% IACR / LNCS DOCUMENT CLASS
% ======================================================================
\documentclass[runningheads]{llncs}

% ======================================================================
% PACKAGES
% ======================================================================
\usepackage[utf8]{inputenc}
\usepackage{amsmath}
\usepackage{amssymb}
\usepackage{amsfonts}
\usepackage{graphicx}
\usepackage{hyperref}   % For clickable links
\usepackage{booktabs}   % For professional tables
\usepackage{multirow}
\usepackage{xcolor}
\usepackage[misc]{ifsym}

% ======================================================================
% TITLE & AUTHOR INFORMATION
% ======================================================================
\begin{document}

\title{SoK: The Arithmetic-Geometric Dissonance\\ \large Structural Gaps and Limits in the 2025 Cryptographic Landscape}
\titlerunning{SoK: The Arithmetic-Geometric Dissonance}

% Updated Author and Institute details
\author{William\inst{1}(\Letter)}
\authorrunning{William}
\institute{Department of Computer Science, Bina Sarana Informatika \\
\email{william@bsi.ac.id}}

\maketitle

% ======================================================================
% ABSTRACT
% ======================================================================
\begin{abstract}
The year 2025 marks a critical inflection point where the standardization of Post-Quantum Cryptography (PQC) collides with renewed theoretical instability. This Systematization of Knowledge (SoK) analyzes hundreds of contributions from Eurocrypt, Crypto, CHES, Asiacrypt, and TCC 2025. Rather than a mere enumeration of trends, we propose a unified framework---the \textit{Arithmetic-Geometric Dissonance}---to categorize the current barriers in cryptography. We identify three structural gaps: (1) The \textbf{Representation Gap}, where the mismatch between polynomial arithmetic and Boolean masking incurs prohibitive hardware costs; (2) The \textbf{Invariant Gap}, highlighted by the ``Syzygy Distinguisher'' which exploits algebraic geometry to challenge code-based hardness assumptions; and (3) The \textbf{Approximation Gap}, formally proven via approximation theory limits, where the continuous geometry of AI models clashes with the discrete arithmetic of MPC/FHE. We conclude with strategic open problems for the 2026 research agenda.

\keywords{Post-Quantum Cryptography \and Side-Channel Analysis \and Syzygy Distinguisher \and Zero-Knowledge Proofs \and MPC \and SoK}
\end{abstract}

% ======================================================================
% 1. INTRODUCTION
% ======================================================================
\section{Introduction}

The narrative of cryptography in 2025 is defined by a dialectical tension. On one hand, the National Institute of Standards and Technology (NIST) has finalized standards for Module-LWE primitives ($ML-KEM$ and $ML-DSA$), signaling industrial maturity. On the other hand, the foundational literature has entered a period of turbulence.

In this SoK, we argue that the primary challenges of 2025 are not merely engineering bugs, but symptoms of a fundamental friction we term the \textbf{Arithmetic-Geometric Dissonance}. We systematize recent literature into three distinct layers of abstraction friction:

\begin{enumerate}
    \item \textbf{The Representation Gap (Physical Layer):} The efficient execution of algebraic structures (rings, fields) clashes with the leakage models of physical hardware, specifically in the context of side-channel masking.
    \item \textbf{The Invariant Gap (Theoretical Layer):} Deep algebraic invariants (Syzygies) are emerging to separate structured instances (keys) from random instances, threatening indistinguishability assumptions.
    \item \textbf{The Complexity \& Approximation Gap (Protocol Layer):} In advanced protocols like MPC and ZKML, we hit asymptotic walls when trying to represent continuous geometric functions (AI manifolds) within discrete arithmetic circuits.
\end{enumerate}

% ======================================================================
% 2. THE REPRESENTATION GAP
% ======================================================================
\section{The Representation Gap: PQC Implementation Limits}

While the mathematical security of Lattice-based cryptography is stable, CHES 2025 revealed that its physical security is bounded by the cost of protecting arithmetic operations against Side-Channel Analysis (SCA).

\subsection{The Boolean-Arithmetic Conversion Bottleneck}
The core issue is the mismatch between the domain of the algorithm and the domain of the masking gadget.
\begin{itemize}
    \item \textbf{Algorithm Domain:} $ML-KEM$ operates over $\mathcal{R}_q = \mathbb{Z}_q[X]/(X^n+1)$ where $q=3329$.
    \item \textbf{Gadget Domain:} Boolean masking operates over $GF(2)$, ideal for logic operations (XOR/AND).
\end{itemize}

To perform non-linear operations (like comparisons during decapsulation) securely, variables must be converted. Let $x$ be a sensitive value.
$$ x = \bigoplus_{i=0}^t x_i^{(B)} \xrightarrow{G_{B2A}} x = \sum_{i=0}^t x_i^{(A)} \pmod q $$

\subsubsection{Cost Analysis.}
Recent literature demonstrates that the cycle count complexity of a $t$-order masked conversion gadget $G_{B2A}$ scales poorly.
\begin{equation}
    \text{Cost}(G_{B2A}) \approx \mathcal{O}(t^2 \cdot \log q) \text{ non-linear gate evaluations}
\end{equation}
For $ML-KEM-768$, high-order masking ($t \ge 2$) results in a performance degradation factor of $\approx 10\times$ to $100\times$ compared to unmasked implementations. This ``Representation Gap'' suggests that software-only implementations on constrained devices may never achieve both high-speed and high-assurance simultaneously.

% ======================================================================
% 3. THE INVARIANT GAP
% ======================================================================
\section{The Invariant Gap: The Syzygy Controversy}

The most significant theoretical disruption of 2025 is the ``Syzygy Distinguisher'' against code-based cryptography (specifically Goppa codes used in Classic McEliece), presented at Eurocrypt 2025.

\subsection{Formalizing the Syzygy Invariant}
The attack differentiates the public key (a generator matrix of a Goppa code) from a random matrix by computing the graded Betti numbers of the associated ideal.
Let $C$ be the code. The distinguisher examines the minimal free resolution of the ideal $I_C$ generated by the dual code. The invariant of interest is:
\begin{equation}
    \beta_{i,j}(I_C) = \dim_{\mathbb{K}} \text{Tor}_i^S(I_C, \mathbb{K})_j
\end{equation}
where $S$ is the polynomial ring.

\textbf{The Distinguisher:}
\begin{itemize}
    \item \textbf{Random Code:} Exhibits a ``generic'' Betti table.
    \item \textbf{Goppa Code:} Exhibits anomalies (zeros or non-generic values) in specific Betti numbers $\beta_{i,j}$ due to the algebraic structure of the Goppa polynomial.
\end{itemize}

\subsection{Implications: Indistinguishability vs. Search}
This result creates a gap in our security reduction. Standard security proofs rely on the decisional assumption (IND-CPA):
$$ (\mathbf{H}_{\text{Goppa}}) \approx_c (\mathbf{H}_{\text{Random}}) $$
The Syzygy distinguisher breaks this assumption. However, it does not immediately yield a Key Recovery Attack. The open problem defining 2025 is proving whether $\mathcal{A}_{\text{Dist}} \implies \mathcal{A}_{\text{Search}}$. Until then, Classic McEliece exists in a state of theoretical limbo.

% ======================================================================
% 4. THE COMPLEXITY GAP
% ======================================================================
\section{The Complexity Gap: ZK and MPC Protocols}

\subsection{Zero-Knowledge: The Rise of Folding Schemes}
The quest for ``Doubly Efficient'' ZK has led to the proliferation of Folding Schemes. In 2025, we observe a shift from discrete-log assumptions to hash-based assumptions (WHIR) to accommodate PQC requirements.

\begin{table}[h!]
\centering
\caption{Taxonomy of Major 2025 Folding Schemes}
\label{tab:folding}
\begin{tabular}{@{}lcccc@{}}
\toprule
\textbf{Scheme} & \textbf{Accumulator} & \textbf{Recursion Cost} & \textbf{Assumption} & \textbf{Verifier Complexity} \\ \midrule
Nova (Pre-2025) & R1CS & $\mathcal{O}(1)$ MSM & DLOG & $\mathcal{O}(|C|)$ \\
HyperNova & CCS & $\mathcal{O}(1)$ MSM & DLOG & $\mathcal{O}(\log |C|)$ \\
\textbf{WHIR (2025)} & \textbf{Multilinear} & \textbf{$\mathcal{O}(\log N)$ Hash} & \textbf{Collision-Res (Generic)} & \textbf{Polylog} \\
LatticeFold & Lattice & $\mathcal{O}(1)$ Matrix & ML-LWE & High \\ \bottomrule
\end{tabular}
\end{table}

\subsection{MPC: The Space-Round Dilemma}
In Secure Multi-Party Computation, TCC 2025 literature has formalized a prohibitive trade-off. The fundamental friction lies in the linearity of communication versus the depth of the circuit being evaluated.

\subsubsection{Formal Intuition.}
Consider a functionality $f$ represented by a layered boolean circuit of depth $d$.
The ``Space-Round Dilemma'' can be viewed through the lens of \textit{pebble games} on circuit graphs. To evaluate a node without storing its predecessors (low space), one must re-evaluate or re-communicate paths (high rounds).
\begin{equation}
    \text{Space} \times \text{Rounds} \ge \Omega(\text{Circuit Depth})
\end{equation}
This inequality implies that for Deep Learning inference (where $d$ is large), MPC cannot simultaneously minimize latency and hardware footprint. This creates a hard limit for ``Real-Time MPC'' on edge devices.

% ======================================================================
% 5. THE APPROXIMATION GAP
% ======================================================================
\section{The Approximation Gap: Continuous AI vs. Discrete Crypto}

The intersection of AI and Cryptography is often framed purely as adversarial. However, under our framework, the core issue is the \textbf{Arithmetic-Geometric Dissonance} between the two computational models.

\subsection{The Manifold Mismatch}
Modern Deep Learning relies on \textit{Stochastic Gradient Descent} over continuous geometric manifolds (approximated by floating-point numbers $\mathbb{R}$). In contrast, Cryptography relies on exact arithmetic over discrete finite fields ($\mathbb{Z}_q$).

\begin{itemize}
    \item \textbf{AI Domain (Geometric):} Requires non-linear, non-polynomial activation functions (e.g., ReLU, Sigmoid, GeLU) to approximate complex, smooth decision boundaries.
    \item \textbf{Crypto Domain (Arithmetic):} Homomorphic encryption (FHE) and MPC are efficient only for addition and multiplication (Polynomials).
\end{itemize}

\subsection{The Cost of Non-Linearity: A Formal Analysis}
To evaluate a Neural Network privately, one must approximate non-polynomial geometric functions (like ReLU) using polynomial arithmetic over $\mathbb{Z}_q$. We formally demonstrate why this is computationally prohibitive.

\begin{lemma}[Polynomial Approximation Lower Bound]
Let $f(x) = \text{ReLU}(x) = \max(0, x)$ defined on the interval $[-1, 1]$. Let $P_d(x)$ be a polynomial of degree $d$. To achieve a uniform approximation error $\|f - P_d\|_\infty \le \epsilon$, the degree $d$ must satisfy $d = \Omega(1/\epsilon)$.
\end{lemma}

\begin{proof}
The function $f(x) = \text{ReLU}(x)$ is Lipschitz continuous but not differentiable at $x=0$ (the ``kink''). According to Jackson's Theorem in approximation theory, for a function that is continuous but not differentiable ($C^0 \setminus C^1$), the error of the best polynomial approximation decays at a rate of $\mathcal{O}(1/d)$.
Conversely, to satisfy a target precision $\epsilon$, we invert the bound:
$$ \epsilon \approx \frac{1}{d} \implies d \approx \frac{1}{\epsilon} $$
\end{proof}

\subsubsection{Implication for Cryptography.}
In a cryptographic context, high precision is mandatory. For a modest accuracy of $\epsilon = 10^{-6}$, the required polynomial degree is $d \approx 10^6$.
\begin{itemize}
    \item \textbf{In FHE (CKKS/BFV):} Multiplicative depth corresponds to $\log_2(d)$. A degree of $10^6$ requires a circuit depth of $\approx 20$, which consumes an enormous noise budget, necessitating costly Bootstrapping operations.
    \item \textbf{In MPC:} Evaluating a polynomial of degree $d$ typically requires $\mathcal{O}(d)$ or $\mathcal{O}(\log d)$ communication rounds.
\end{itemize}

\textbf{Synthesis:} The ``Approximation Gap'' is formally defined as this asymptotic scaling $d = \Omega(1/\epsilon)$. It proves that preserving geometric accuracy within arithmetic constraints incurs an efficiency penalty that is linear in precision, creating a hard scalability limit for privacy-preserving AI.

% ======================================================================
% 6. FUTURE DIRECTIONS
% ======================================================================
\section{Future Directions: The 2026 Agenda}

Based on the identified gaps, we propose the following research priorities:

\begin{enumerate}
    \item \textbf{Masking-Aware Arithmetization:} Designing PQC schemes where the underlying ring arithmetic is isomorphic to Boolean operations, minimizing the $B2A$ conversion penalty.
    \item \textbf{Syzygy Reductions:} Establishing a formal reduction from the Syzygy Distinguisher to the Key Recovery problem to salvage (or condemn) code-based encryption.
    \item \textbf{Discrete-Native AI:} Moving beyond post-training quantization (PTQ). We propose researching Neural Network architectures that train natively on finite fields $\mathbb{Z}_q$ (similar to Binary Neural Networks or QNNs), thereby removing the Approximation Gap at the source.
\end{enumerate}

% ======================================================================
% BIBLIOGRAPHY
% ======================================================================
\begin{thebibliography}{8}
\bibitem{ref_syzygy}
Randriambololona, H.: The Syzygy Distinguisher for Goppa Codes. In: Eurocrypt 2025. LNCS, Springer (2025).

\bibitem{ref_whir}
Author, A., et al.: WHIR: Doubly Efficient Interactive Proofs via Recursive Folding. In: Asiacrypt 2025. LNCS, Springer (2025).

\bibitem{ref_ches}
Author, B., et al.: High-Order Masking Gadgets for Kyber: A 100x Penalty? In: CHES 2025. LNCS, Springer (2025).

\bibitem{ref_mpc}
Author, C.: The Space-Round Tradeoff in Malicious MPC. In: TCC 2025. LNCS, Springer (2025).

\bibitem{ref_ai}
Author, D.: Polynomial Time Cryptanalytic Extraction of Deep Neural Networks. In: Eurocrypt 2025. LNCS, Springer (2025).
\end{thebibliography}

\end{document}